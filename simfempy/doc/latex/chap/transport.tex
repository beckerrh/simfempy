% !TEX root = ../simfempy.tex
%
%==========================================
\section{Discreization of the transport equation}\label{sec:}
%==========================================
%
For $k\in\N_0$ we denote by $\Cspace_h^k(\Omega)$ the space of piecewise $k$-times differential functions with respect to $\Cells_h$, and 
piecewise differential operators $\nabla_h:\Cspace_h^l(\Omega)\to \Cspace_h^{l-1}(\Omega,\R^d)$ ($l\in\N$) by 
$\Rest{\nabla_h q}{K} := \nabla\left(\Rest{q}{K}\right)$ for $q\in \Cspace_h^l(\Omega)$ and similarly for 
$\div_h: \Cspace_h^{l}(\Omega,\R^d)\to \Cspace_h^{l-1}(\Omega)$.
We frequently use the piecewise Stokes formula
%
\begin{equation}\label{eq:StokesPiecewise}
\int_{\Omega} (\nabla_h q) v + \int_{\Omega} q (\div_h v) = \int_{\SidesInt_h}\jump{q v_n} +  \int_{\SidesBdry_h}q v_n,
\end{equation}
%
where $\int_{\Sides_h}=\sum_{S\in\Sides_h}\int_S$ and $n$ in the sum stands for $n_S$.

The subspace of piecewise polynomial functions of order $k\in\N_0$ in $C^k_h(\Omega)$ is denoted by $\Dspace_h^k(\Omega)$ and the $L^2(\Omega)$-projection by $\pi_h^k:L^2(\Omega)\to\Dspace_h^k(\Omega)$.


Suppose $u$ satisfies
%
\begin{equation}\label{eq:transport}
\div(\beta u) = f\quad\mbox{in $\Omega$},\qquad \beta_n^{-}(u-\uD) = 0\quad \mbox{on $\partial\Omega$}.
\end{equation}
%
From the integration by parts formula
%
\begin{equation}\label{eq:}
\int_{\Omega}\div(\beta u ) v  = - \int_{\Omega}\beta u \cdot \nabla v + \int_{\partial\Omega} \beta_n uv
\end{equation}
%
it then follows that $u$ satisfies 
%
\begin{align*}
a(u,v)=l(v)\quad \forall v\in V
\end{align*}
%
with
%
\begin{equation}\label{eq:}
a(u,v) := -\int_{\Omega} u \beta\cdot \nabla v + \int_{\partial\Omega} \beta_n^+ uv,\quad l(v) := \int_{\Omega} fv -\int_{\partial\Omega} \beta_n^- \uD v.
\end{equation}
%
We also have
%
\begin{align*}
a(u,v) =& -\frac12\int_{\Omega} u  \beta\cdot\nabla v + \frac12 \int_{\Omega}\div(\beta u ) v - \frac12 \int_{\partial\Omega} \beta_n uv +\int_{\partial\Omega}\beta_n^+ uv\\
=&  \frac12 \int_{\Omega}\div(\beta) u  v +\frac12\left(\int_{\Omega} \beta\cdot\nabla u v-\int_{\Omega} u  \beta\cdot\nabla v\right)
+ \frac12 \int_{\partial\Omega}\abs{\beta_n}uv
\end{align*}
%

%
%-------------------------------------------------------------------------
\subsection{$\Pspace^1_h(\Omega)$}\label{subsec:}
%-------------------------------------------------------------------------
%
%
Let $K\in\Cells_h$, $\beta_K=\pi_K\beta$, $x_K$ be the barycenter of $K$ and $\xup\in\partial K$ such that  with 
$\delta_K\ge0$
%
\begin{equation}\label{eq:}
\xup = x_K + \delta_K \beta_K
\end{equation}
%
If we know $\transpose{\normal[i]}\beta_K$, we can compute $\xup$ as follows.
%
\begin{align*}
\lambda_i(\xup) = \lambda_i(x_K) + \delta_K \transpose{\nabla\lambda_i}\beta_K
= \frac{1}{d+1} - \delta_K \frac{\transpose{\normal[i]}\beta_K}{h_i}
= \frac{1}{d+1} - \delta_K \frac{\transpose{\normal[i]}\beta_K \abs{S_i}}{d\abs{K}}
\end{align*}
%
It follows that
%
\begin{equation}\label{eq:}
\delta_K = \max\SetDef{\frac{d\abs{K}}{(d+1)\abs{S_i}\left(\transpose{\normal[i]}\beta_K\right)^+}}{0\le i\le d}.
\end{equation}
%
The stabilized bilinear form is
%
\begin{equation}\label{eq:}
a^{\rm supg}(u,v) := \int_{\Omega} (\beta\cdot \nabla  u) v - \int_{\partial\Omega} \beta_n^- uv + \int_{\Omega} \delta(\beta\cdot \nabla u)( \beta\cdot \nabla v)
\end{equation}
%
Then we have
%
\begin{align*}
a^{\rm supg}(u,v) = 
\end{align*}
%



%==========================================
\printbibliography[title=References Section~\thesection]
%==========================================

