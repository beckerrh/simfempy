% !TEX root = ../simfempy.tex
%
%==========================================
\section{Stokes problem}\label{sec:}
%==========================================
%
Let $\Omega\subset \R^d$, $d=2,3$ be the computational domain. We suppose to have a disjoined partition of its boundary:
$\partial\Omega=\GammaD\cup\GammaN\cup\GammaR\cup\GammaP$, the outward unit normal is denotes by $n$. We denote by 
$\pitang := (I-n\transpose{n})\in \R^{(d-1)\times d}$ the projection on the tangent plane.
%
\begin{equation}\label{eq:stokes:strongform}
%
\left\{
\begin{aligned}
-\div\left(\mu\nabla v\right) + \nabla p = f \quad \mbox{in $\Omega$}\\
\div v  = g \quad \mbox{in $\Omega$}\\
v = \vD \quad \mbox{in $\GammaD$},\\
%\mu\frac{\partial v}{\partial n} - p n = -\pN n\quad \mbox{in $\GammaN$}\\
%
\left\{
\begin{aligned}
\pitang \mu \frac{\partial v}{\partial n}  =& 0\\
 \mu \frac{\partial v}{\partial n}\cdot n -p   =& -\pN
\end{aligned}
\right\}\quad \mbox{in $\GammaN$},\\
%
%
\left\{
\begin{aligned}
v_n =& \vRn\\
\pitang \left( \lambda_R v + \mu \frac{\partial v}{\partial n}\right)  =& \pitang\gR 
\end{aligned}
\right\}
%
\quad \mbox{in $\GammaR$},\\
%
\left\{
\begin{aligned}
\pitang v =& \pitang \vP\\
\left(  \mu \frac{\partial v}{\partial n}\cdot n -p \right)  =& -\pP
\end{aligned}
\right\}
%
\quad \mbox{in $\GammaP$}.
\end{aligned}
\right.
%
\end{equation}
%
We can express the equations by means of the Cauchy stress tensor
%
\begin{equation}\label{eq:Stokes:CauchyStress}
\sigma := 2\mu D(v) + \lambda\div(v) I - p I,\quad  D(v)  = \frac12 \left(\nabla v + \transpose{\nabla v}\right).
\end{equation}
%
Due to $\div v=0$, the bulk viscosity $\lambda$ is neglected.

Then the momentum balance is (with the row-wise divergence operator)
%
\begin{align*}
-\div \sigma = f \quad \mbox{in $\Omega$}.
\end{align*}
%
The weak formulation (\ref{eq:stokes:weakform}) is based on the non-symmetric
%
\begin{equation}\label{eq:Stokes:CauchyStressnonsymmetric}
\widetilde{\sigma} := \mu \nabla v - p I,
\end{equation}
%
which is equivalent to using $\sigma$ for volume integrals since $A:B=\frac{A+\transpose{A}}{2}:B$ for all symmetric $B\in\R^{d\times d}$ and any $A\in\R^{d\times d}$.

Using $\sigma$ in a weak formulation will in general generate different boundary conditions.
%
%-------------------------------------------------------------------------
\subsection{Weak formulation}\label{subsec:}
%-------------------------------------------------------------------------
%
The standard weak formulation reads
%
\begin{equation}\label{eq:stokes:weakform}
%
\left\{
\begin{aligned}
&V_{\vD,\vRn,\vP} :=  \SetDef{v\in H^1(\Omega,\R^d)}{\Rest{v}{\Gamma_D}=\vD\;\&\; \Rest{v_n}{\Gamma_R}\;\&\; \Rest{\pitang v}{\Gamma_P}=\pi\vP}\\&Q := L^2(\Omega)
\quad \left( Q := L^2(\Omega)/\R \quad\mbox{if $\abs{\GammaN\cup\GammaP}=0$}\right)\\
&(v,p)\in V_{\vD,\vRn,\vP}\times Q:\quad a_{\Omega}(v,p; \phi\xi)  = l_{\Omega}(\phi,\xi) \quad\forall (\phi,\xi)\in V_{0,0,0}\times Q\\
&\begin{aligned}
a_{\Omega}(v,p; \phi,\xi) &:= \int_{\Omega} \mu \nabla v:\nabla\phi - \int_{\Omega}p \div \phi + \int_{\Omega} \div v \xi
+ \lambda_R\int_{\Gamma_R}\left(v\cdot\phi - v_n\phi_n\right),\\
 l_{\Omega}(\phi,\xi) &:= \int_{\Omega} f\cdot\phi + \int_{\Omega} g\xi + \int_{\GammaR}\left(\gR\cdot\phi - \gR_n\phi_n\right)
 -\int_{\GammaN}\pN \phi_n -\int_{\GammaP}\pP \phi_n.
\end{aligned}
\end{aligned}
\right.
%
\end{equation}
%
%---------------------------------------
\begin{lemma}\label{lemma:}
A regular solution of the formulation (\ref{eq:stokes:weakform}) satisfies (\ref{eq:stokes:strongform}).
\end{lemma}
%
%---------------------------------------
\begin{proof}
By integration by parts we have, together with $v\cdot\phi - v_n\phi_n=\pitang v\cdot\phi$
%
\begin{align*}
a_{\Omega}(v,p; \phi,\xi) = \int_{\Omega} \left( -\mu \Delta v+\nabla p\right)\cdot\phi + 
\int_{\partial\Omega} \mu\dn{v}\cdot\phi - \int_{\partial\Omega} p\phi_n
+\int_{\Omega} \div v \xi+ \lambda_R\int_{\Gamma_R}\pitang v\cdot\phi
\end{align*}
%
Then the (regular) weak solution satisfies for all $\phi$
%
\begin{align*}
\int_{\Omega} \left( -\mu \Delta v+\nabla p -f\right)\cdot\phi
=& \int_{\partial\Omega} p\phi_n - \int_{\partial\Omega} \mu\dn{v}\cdot\phi -\int_{\GammaN} \pN\phi_n-\int_{\GammaP} \pP\phi_n+ \int_{\GammaR} \pitang(\gR -\lambda_R v)\cdot\phi,
\end{align*}
%
and for all $\xi$
%
\begin{align*}
\int_{\Omega} (\div v-g) \xi = 0.
\end{align*}
%

Taking $\phi \in H^1_0(\Omega,\R^d)$, the right-hand side vanishes and the density of this space in $L^2(\Omega)$ gives us 
%
\begin{align*}
-\mu \Delta v+\nabla p = f,\quad \div v = g \quad  \mbox{a.e. in $\Omega$}.
\end{align*}
%
But this means that for general $\phi\in V_{0,0,0}$
%
\begin{align*}
\int_{\partial\Omega} p\phi_n - \int_{\partial\Omega} \mu\dn{v}\cdot\phi -\int_{\GammaN} \pN\phi_n -\int_{\GammaP} \pP\phi_n
+ \int_{\GammaR} \pitang(\gR -\lambda_R v)\cdot\phi=0
\end{align*}
%
Decomposing the test function as
%
\begin{align*}
\phi = \phi_n n + \pitang\phi
\end{align*}
%
and using the definition of $V_{0,0,0}$ we find
%
\begin{align*}
\int_{\Gamma_N} \left( (p-\pN)n - \mu\dn{v}\right)\cdot\phi 
+ \int_{\Gamma_P}  (p-\pP- \mu\dn{v}\cdot n)\phi_n 
+ \int_{\Gamma_R} \pitang(\gR -\lambda_R v - \mu\dn{v})\cdot\phi =0
\end{align*}
%
\end{proof}
%
%---------------------------------------
\begin{proposition}\label{prop:}
If we use the weak formulation based on the stress tensor
%
\begin{equation}\label{eq:}
a_{\Omega}(v,p; \phi,\xi) := \int_{\Omega} \sigma:\nabla\phi + \int_{\Omega} \div v \xi
+ \lambda_R\int_{\Gamma_R}\pitang v\cdot\pitang \phi,
\end{equation}
%
the resulting boundary conditions are 
%
\begin{equation}\label{eq:stokes:bccauchy}
%
\begin{aligned}
%
\left\{
\begin{aligned}
v = \vD \quad &\mbox{in $\GammaD$},\\
\sigma n= -\pN n\quad &\mbox{in $\GammaN$},\\
%%
\left\{
\begin{aligned}
\pitang v =& \pitang \vP\\
(I-\pitang) \sigma n  =& -\pP 
\end{aligned}
\right\}
\quad  &\mbox{in $\GammaP$},\\
%%
\left\{
\begin{aligned}
v_n =& \vRn\\
\pitang \left( \lambda_R v + \sigma n\right)  =& \pitang\gR 
\end{aligned}
\right\}
%%
\quad &\mbox{in $\GammaR$}.
\end{aligned}
\right.
%
\quad\Leftrightarrow\quad
\left\{
\begin{aligned}
v = \vD \quad &\mbox{in $\GammaD$},\\
\sigma n= -\pN n\quad &\mbox{in $\GammaN$},\\
%%
\left\{
\begin{aligned}
\pitang \mu \frac{\partial v}{\partial n} - \mu\omega n  =& 0\\
 \mu \frac{\partial v}{\partial n}\cdot n -p   =& -\pN
\end{aligned}
\right\}
\quad &\mbox{in $\GammaN$},\\
%%
%%
\left\{
\begin{aligned}
\pitang v =& \pitang \vP\\
 \mu \frac{\partial v}{\partial n}\cdot n -p   =& -\pP
\end{aligned}
\right\}
\quad &\mbox{in $\GammaP$},\\
%%
\left\{
\begin{aligned}
v_n =& \vRn\\
\pitang \left( \lambda_R v +\mu \dn{v} - \mu\omega n\right)  =& \pitang\gR 
\end{aligned}
\right\}
%%
\quad &\mbox{in $\GammaR$}.
\end{aligned}
\right.
%
\end{aligned}
\end{equation}
%
\end{proposition}
%
%---------------------------------------
\begin{proof}
Using now
%
%
\begin{equation}\label{eq:Stokes:ippsigma}
\int_{\Omega} \sigma:\nabla\phi = -\int_{\Omega} \div \sigma\cdot \phi+ \int_{\partial \Omega} \sigma n\cdot \phi
\end{equation}
%
we get in similar way as before
%
\begin{align*}
\int_{\Gamma_N} \left( \pN n + \sigma n\right)\cdot\phi 
+ \int_{\Gamma_P}  (I-\pitang)\left( \pP n+ \sigma n\right)\cdot\phi
+ \int_{\Gamma_R} \pitang\left ( \lambda_R v + \sigma n -\gR \right)\cdot\phi =0
\end{align*}
%
%%
%\begin{align*}
%-\int_{\Gamma_N} \left( \sigma n + \pN n \right)\cdot\phi 
%%+ \int_{\Gamma_R}\left(\gortho - \lambda_Rv_{\northo} - \sigma n \cdot \northo \right) \phi_{\northo}=0
%\end{align*}
%%
We have 
%
\begin{align*}
\transpose{n}\transpose{(\nabla v)} n =& \transpose{(\nabla v n)} n = \transpose{\dn{v}} n= \transpose{n}\dn{v}=
\transpose{n}(\nabla v) n
\end{align*}
%
and therefore
%
\begin{align*}
\transpose{n}\sigma n =& \transpose{n}\widetilde{\sigma} n\quad\Rightarrow\quad (I-\pitang)\sigma n= (I-\pitang)\widetilde{\sigma} n,\\ 
\pitang \sigma n =& (I-n\transpose{n})\sigma n = \sigma n - n\transpose{n}\widetilde{\sigma} n = \pitang \widetilde{\sigma} n  + (\sigma-\widetilde{\sigma}) n = \pitang \widetilde{\sigma} n  - \mu \omega n
\end{align*}
%
with $\omega := \frac{1}{2}(\nabla v-\transpose{\nabla v})$.
\end{proof}
%
%
%-------------------------------------------------------------------------
\subsection{Discretization}\label{subsec:}
%-------------------------------------------------------------------------
%
We use finite element spaces $V_h$ for the velocity and $Q_h$ for the pressure. One main difficulty is to obtain a stable approximation of the pressure gradient, which requires the inf-sup condition
%
\begin{equation}\label{eq:stokes:infsupgrad}
\inf_{p\in Q_h\setminus\Set{0}}\sup_{v\in V_h\setminus\Set{0}} \frac{\int_{\Omega}p\div v}{\norm{v}_V\norm{p}_Q} \ge \gamma >0.
\end{equation}
%
To this end, we use the classical spaces $V_h = \CR^1_h(\Omega,\R^d)$ and $Q_h=\Dspace_h^0$.
%
%-------------------------------------------------------------------------
\subsection{Implementations of Boundary condition}\label{subsec:}
%-------------------------------------------------------------------------
%
%
%~~~~~~~~~~~~~~~~~~~~~~~~~~~~~
\subsubsection{Strong implementation of Dirichlet condition}
%~~~~~~~~~~~~~~~~~~~~~~~~~~~~~
%
We write the discrete velocity space $V_h$ as a direct sum $V_h = \Vint_h \oplus \Vdir_h$, with $\Vdir_h$ corresponding to the discrete functions not vanishing on $\GammaD$. 
Splitting the matrix and right-hand side vector correspondingly, and letting $u^D_h\in \Vdir_h$ be an approximation of the Dirichlet data $\vD$ we have the traditional way to implement Dirichlet boundary conditions:
%
\begin{equation}\label{eq:StokesDirTrad}
\begin{bmatrix}
\bdryint{A} & 0 & -\transpose{\bdryint{B}}\\
0 & I & 0 \\
\bdryint{B} & 0 & 0
\end{bmatrix}
\begin{bmatrix}
\bdryint{v}_h\\
\bdrydir{v}_h\\
p_h
\end{bmatrix}
=
\begin{bmatrix}
\bdryint{f} - \Aintdir v^D_h\\
v^D_h\\
g - \bdrydir{B} v^D_h
\end{bmatrix}.
\end{equation}
%
As for the Poisson problem, we obtain an alternative formulation   
%
\begin{equation}\label{eq:StokesDirNew}
\begin{bmatrix}
\bdryint{A} & 0 & -\transpose{\bdryint{B}}\\
0 & \bdrydir{A} & 0 \\
\bdryint{B} & 0 & 0
\end{bmatrix}
\begin{bmatrix}
\bdryint{v}_h\\
\bdrydir{v}_h\\
p_h
\end{bmatrix}
=
\begin{bmatrix}
\bdryint{f} - \Aintdir v^D_h\\
\bdrydir{A} v^D_h\\
g - \bdrydir{B} v^D_h
\end{bmatrix}.
\end{equation}
%
%
%~~~~~~~~~~~~~~~~~~~~~~~~~~~~~
\subsubsection{Weak implementation (Nitsche's method)}
%~~~~~~~~~~~~~~~~~~~~~~~~~~~~~
%
%
Instead of modifying the discrete velocity space, we add additional terms to the bilinear and linear forms.
%
\begin{equation}\label{eq:stokes:nitsche}
%
\left\{
\begin{aligned}
&(v,p)\in V_h\times Q_h:\quad a_{\Omega}(v,p; \phi\xi) + a_{\partial\Omega}(v,p; \phi,\xi) = l_{\Omega}(\phi,\xi)+l_{\partial\Omega}(\phi,\xi) \quad\forall (\phi,\xi)\in V_h\times Q_h\\
&\begin{aligned}
 a_{\partial\Omega}(v,p; \phi,\xi) &:= \int_{\GammaD\cup\GammaP}\frac{\gamma\mu}{h}v\cdot\phi - 
 \int_{\GammaD\cup\GammaP}\mu \left(  \dn{v}\cdot  \phi + v  \cdot \dn{\phi}\right)\\
 &+\int_{\GammaR}\frac{\gamma\mu}{h}v_n\phi_n - \int_{\GammaR} \mu\left( \dn{v}\cdot n \phi_n + v_n\dn{\phi}\cdot n \right)\\
 &-\int_{\GammaP}\frac{\gamma\mu}{h}v_n\phi_n + \int_{\GammaP} \mu\left( \dn{v}\cdot n \phi_n + v_n\dn{\phi}\cdot n \right)\\
%& + \int_{\GammaP}\frac{\gamma\mu}{h}\pitang v\cdot\pitang\phi
%- \int_{\GammaP} \mu\left( \pitang\dn{v}\cdot\pitang\phi + \pitang v\cdot\pitang\dn{\phi}\right)\\
&+ \int_{\GammaD\cup\GammaR} \left( p\phi_n - v_n\xi\right)
 \\
 l_{\partial\Omega}(\phi,\xi) =& \int_{\GammaD}\mu\vD\cdot\left( \frac{\gamma}{h}\phi - \dn{\phi}\right)- \int_{\GammaD}\vD_n\xi
+ \int_{\GammaR}\mu\vRn\cdot\left( \frac{\gamma}{h}\phi_n - \dn{\phi}\cdot n\right)- \int_{\GammaR}\vRn\xi\\
&+ \int_{\GammaP}\mu\pitang\vP\cdot\pitang\left( \frac{\gamma}{h}\phi - \dn{\phi}\right).
\end{aligned}
\end{aligned}
\right.
%
\end{equation}
%
%
%---------------------------------------
\begin{lemma}\label{lemma:}
A regular continuous solution of the formulation (\ref{eq:stokes:weakform}) satisfies (\ref{eq:stokes:nitsche}).
\end{lemma}
%
%---------------------------------------
\begin{proof}
We have already seen that a regular continuous solution satisfies for $(\phi,\xi)\in V_h\times Q_h$
%
%
\begin{align*}
a_{\Omega}(v,p; \phi,\xi)  - l_{\Omega}(\phi,\xi) =& 
\int_{\GammaD} \left( \mu\dn{v} - pn\right)\cdot\phi
+ \int_{\GammaR} \left( \mu\dn{v}\cdot n-p\right)\phi_n
+ \int_{\GammaP} \pitang\mu\dn{v} \cdot\phi
\end{align*}
%
Thanks to the boundary conditions we also have
%
\begin{align*}
\int_{\GammaD}\mu(\vD-v)\cdot\left( \frac{\gamma}{h}\phi - \dn{\phi}\right)- \int_{\GammaD}(\vD_n-v_n)\xi =0\\
\int_{\GammaR} \mu(\vRn-v_n)\left( \frac{\gamma}{h}\phi_n - \dn{\phi}\cdot n\right) - \int_{\GammaR}(\vRn-v_n)\xi = 0\\
\int_{\GammaP}\mu\pitang(\vP-v)\cdot\left( \frac{\gamma}{h}\phi - \dn{\phi}\right) =0
\end{align*}
%
Adding these terms we get
%
\begin{align*}
a_{\Omega}(v,p; \phi\xi)  - l_{\Omega}(\phi,\xi) =&
-\int_{\GammaD}\frac{\gamma\mu}{h}v\cdot\phi
+ \int_{\GammaD} \mu\left( \dn{v}\cdot\phi + v\cdot\dn{\phi} \right)
- \int_{\GammaD} \left( p\phi_n - v_n\xi\right)\\
&+ \int_{\GammaD}\mu\vD\cdot\left( \frac{\gamma}{h}\phi - \dn{\phi}\right)- \int_{\GammaD}\vD_n\xi\\
&-\int_{\GammaR}\frac{\gamma\mu}{h}v_n\phi_n
+ \int_{\GammaR} \mu\left( \dn{v}\cdot n \phi_n + v_n\dn{\phi}\cdot n \right)
- \int_{\GammaR} \left( p\phi_n - v_n\xi\right)\\
&+ \int_{\GammaR}\mu\vRn\cdot\left( \frac{\gamma}{h}\phi_n - \dn{\phi}\cdot n\right)- \int_{\GammaR}\vRn\xi\\
&-\int_{\GammaP}\frac{\gamma\mu}{h}\pitang v\cdot\pitang\phi
+ \int_{\GammaP} \mu\left( \pitang\dn{v}\cdot\pitang\phi + \pitang v\cdot\pitang\dn{\phi} \right)\\
&+ \int_{\GammaP}\mu\pitang\vP\cdot\pitang\left( \frac{\gamma}{h}\phi - \dn{\phi}\right)\\
%
=& l_{\partial\Omega}(\phi,\xi) - a_{\partial\Omega}(v,p; \phi,\xi)
\end{align*}
%

%
\end{proof}
%



Alternatively, we can write the system as
%
\begin{equation}\label{eq:stokes:nitsche2}
%
\left\{
\begin{aligned}
&(v,p)\in V_h\times Q_h:\quad a(v,p; \phi\xi) + b(v, \xi)- b(\phi, p) = l_{\Omega}(\phi,\xi)+l_{\partial\Omega}(\phi,\xi) \quad\forall (\phi,\xi)\in V_h\times Q_h\\
&\begin{aligned}
a(v,p; \phi,\xi) &:= \int_{\Omega} \mu \nabla v:\nabla\phi + \lambda_R\int_{\Gamma_R}\left(v\cdot\phi - v_n\phi_n\right)+\int_{\GammaD\cup\GammaP}\frac{\gamma\mu}{h}v\cdot\phi - 
 \int_{\GammaD\cup\GammaP}\mu \left(  \dn{v}\cdot n \phi + v n \cdot \dn{\phi}\right)\\
 &+\int_{\GammaR}\frac{\gamma\mu}{h}v_n\phi_n - \int_{\GammaR} \mu\left( \dn{v}\cdot n \phi_n + v_n\dn{\phi}\cdot n \right)
 -\int_{\GammaP}\frac{\gamma\mu}{h}v_n\phi_n + \int_{\GammaP} \mu\left( \dn{v}\cdot n \phi_n + v_n\dn{\phi}\cdot n \right)\\
b(v, \xi) &:=  \int_{\Omega} \div v \xi - \int_{\GammaD} v_n \xi
\end{aligned}
\end{aligned}
\right.
%
\end{equation}
%
%
%-------------------------------------------------------------------------
\subsection{Computation of boundary forces}\label{subsec:}
%-------------------------------------------------------------------------
%
Suppose $\psi\in\R^d$ is a vector field wich equals $\vec{d}\in\R^d$ on a given part $\Gamma\subset\partial\Omega$ of the boundary and vanishes on its complement. Then we get the sum of the forces on $\Gamma$ in direction $\vec{d}$ by means of the integration by parts formula (\ref{eq:Stokes:ippsigma}), supposed the weak solution is sufficiently smooth, as
%
\begin{align*}
\int_{\Gamma} \transpose{n}\sigma \vec{d} =& \int_{\partial\Omega} \transpose{n}\sigma \psi
=\int_{\Omega} \sigma:\nabla\psi  + \int_{\Omega} \div \sigma\cdot \psi
=\int_{\Omega} \sigma:\nabla\psi  -\int_{\Omega} f\cdot \psi\\
=& a_{\Omega}(v,p; \psi,0) - l_{\Omega}(\psi,0) - \lambda_R\int_{\Gamma_R}\left(v\cdot\psi - v_n\psi_n\right)+ \int_{\GammaR}\left(\gR\cdot\psi - \gR_n\psi_n\right)
 -\int_{\GammaN}\pN \psi_n -\int_{\GammaP}\pP \psi_n
\end{align*}
%
Supposing $\psi$ is a discrete vector field (in general we have to approximate it), for the strong implementation, we can retrieve the last expression by the parts of the matrix eliminated.
Fo the weak implementation we have, since $\psi$ is now an admissible test function
%
\begin{align*}
\int_{\Gamma} \transpose{n}\sigma \vec{d} =&a_{\Omega}(v,p; \psi,0) - l_{\Omega}(\psi,0) - \lambda_R\int_{\Gamma_R}\left(v\cdot\psi - v_n\psi_n\right)+ \int_{\GammaR}\left(\gR\cdot\psi - \gR_n\psi_n\right)
 -\int_{\GammaN}\pN \psi_n-\int_{\GammaP}\pP \psi_n\\
=&  l_{\partial\Omega}(\psi,0) - a_{\partial\Omega}(v,p; \psi,0) - \lambda_R\int_{\Gamma_R}\left(v\cdot\psi - v_n\psi_n\right)+ \int_{\GammaR}\left(\gR\cdot\psi - \gR_n\psi_n\right)
 -\int_{\GammaN}\pN \psi_n-\int_{\GammaP}\pP \psi_n\\
 =&
 \int_{\GammaD}\mu\vD\cdot \frac{\gamma}{h}\psi 
+ \int_{\GammaR}\mu\vRn\cdot\frac{\gamma}{h}\psi_n
+ \int_{\GammaP}\mu\pitang\vP\cdot\pitang \frac{\gamma}{h}\psi \\
&- \int_{\GammaD\cup\GammaP}\frac{\gamma\mu}{h}v\cdot\psi + 
 \int_{\GammaD\cup\GammaP}\mu  \dn{v}\cdot  \psi \\
 &-\int_{\GammaR}\frac{\gamma\mu}{h}v_n\psi_n - \int_{\GammaR} \mu \dn{v}\cdot n \psi_n \\
 &+\int_{\GammaP}\frac{\gamma\mu}{h}v_n\psi_n + \int_{\GammaP} \mu \dn{v}\cdot n \psi_n \\
 &- \lambda_R\int_{\Gamma_R}\left(v\cdot\psi - v_n\psi_n\right)+ \int_{\GammaR}\left(\gR\cdot\psi - \gR_n\psi_n\right)
 -\int_{\GammaN}\pN \psi_n-\int_{\GammaP}\pP \psi_n
\end{align*}
%
%
\begin{equation}\label{eq:}
%
\int_{\Gamma} \transpose{n}\sigma \vec{d} =
\left\{
\begin{aligned}
& 
 \int_{\Gamma} \left(  \mu\dn{v} -p n  \right)\cdot  d
+\int_{\Gamma}\frac{\gamma\mu}{h}(\vD-v)\cdot d
&\quad \Gamma\subset\GammaD\\ 
& 
 \int_{\Gamma} \mu\left( \dn{v}\cdot n \phi_n \right)
- \int_{\Gamma} p\phi_n  
+\int_{\Gamma}\mu(\vRn-v_n)\cdot\left( \frac{\gamma}{h}\phi_n - \dn{\phi}\cdot n\right)
&\quad \Gamma\subset\GammaR
\end{aligned}
\right.
%
\end{equation}
% 
%
%-------------------------------------------------------------------------
\subsection{Pressure mean}\label{subsec:}
%-------------------------------------------------------------------------
%
If no boundary conditions is of Neumann or Pressure type, the pressure is only determined up to a constant. In order to impose the zero mean on the pressure, let $C$ the matrix of size $(1,nc)$
%
\begin{equation}\label{eq:StokesDirPmean}
\begin{bmatrix}
A  & -\transpose{B}& 0\\
B & 0 & \transpose{C}\\
0 & C & 0
\end{bmatrix}
\begin{bmatrix}
v\\
p\\
\lambda
\end{bmatrix}
=
\begin{bmatrix}
f\\
g\\
0
\end{bmatrix}.
\end{equation}
%
Let us considered solution of (\ref{eq:StokesDirPmean}) with $S=BA^{-1}\transpose{B}$, $T=CS^{-1}\transpose{C}$
%
\begin{equation}\label{eq:}
%
\left\{
\begin{aligned}
&A \tilde v &&= f\\
&S \tilde p &&= g-B\tilde v\\
&T \lambda &&= -C\tilde p\\
&S (p-\tilde p) &&= \transpose{C} \lambda\\
&A(v-\tilde v) &&=  \transpose{B} p
\end{aligned}
\right.
%
\end{equation}
%
%
%-------------------------------------------------------------------------
\subsection{Iterative solution}\label{subsec:}
%-------------------------------------------------------------------------
%
%
\begin{align*}
\mathcal A = 
\begin{bmatrix}
A & -\transpose{B}\\
B & 0 
\end{bmatrix}
=
\begin{bmatrix}
I  & 0\\
BA^{-1} & I
\end{bmatrix}
\begin{bmatrix}
A  & -\transpose{B}\\
0 & S
\end{bmatrix}
\quad
\mathcal A^{-1} = 
\begin{bmatrix}
A^{-1}  & A^{-1}\transpose{B}S^{-1}\\
0 & S^{-1}
\end{bmatrix}
\begin{bmatrix}
I  & 0\\
-BA^{-1} & I
\end{bmatrix}
\quad
S = B A^{-1}\transpose{B}
\end{align*}
%
This means that the solution of $\mathcal A (x_v,x_p) = (y_v,y_p)$ is given by
%
\begin{align*}
%
\left\{
\begin{aligned}
A {x_v'} =& y_v\\
S {x_p'} =& y_p - B {x_v'}\\
A x_v =& y_v + \transpose{B} x_p'\\
\end{aligned}
\right.
%
\end{align*}
%
%
%~~~~~~~~~~~~~~~~~~~~~~~~~~~~~
\subsubsection{What is the effect of replacing $S$ by $K$?}
%~~~~~~~~~~~~~~~~~~~~~~~~~~~~~
%
Let
%
\begin{align*}
\mathcal B^{-1} := 
\begin{bmatrix}
A^{-1}  & A^{-1}\transpose{B}K^{-1}\\
0 & K^{-1}
\end{bmatrix}
\begin{bmatrix}
I  & 0\\
-BA^{-1} & I
\end{bmatrix}
=
\begin{bmatrix}
A^{-1}(I-\transpose{B}K^{-1}BA^{-1})  & A^{-1}\transpose{B}K^{-1}\\
-K^{-1}BA^{-1} & K^{-1}
\end{bmatrix}
\end{align*}
%
Then we have
%
\begin{align*}
\mathcal A\mathcal B^{-1} = 
\begin{bmatrix}
I &  0\\
(I-SK^{-1})BA^{-1}& SK^{-1}
\end{bmatrix}
\end{align*}
%


%
%~~~~~~~~~~~~~~~~~~~~~~~~~~~~~
\subsubsection{What is the effect of replacing $A$ by $X$?}
%~~~~~~~~~~~~~~~~~~~~~~~~~~~~~
%
Let
%
\begin{align*}
\mathcal M := \begin{bmatrix}
X & -\transpose{B}\\
B & 0 
\end{bmatrix}
\end{align*}
%

Then with $T=B X^{-1}\transpose{B}$
%
\begin{align*}
\mathcal M^{-1}\mathcal A =&
\begin{bmatrix}
X^{-1}  & X^{-1}\transpose{B}T^{-1}\\
0 & T^{-1}
\end{bmatrix}
\begin{bmatrix}
I  & 0\\
-BX^{-1} & I
\end{bmatrix}
\begin{bmatrix}
A & -\transpose{B}\\
B & 0 
\end{bmatrix}\\
=&
%\begin{bmatrix}
%X^{-1}  & X^{-1}\transpose{B}T^{-1}\\
%0 & T^{-1}
%\end{bmatrix}
%\begin{bmatrix}
%I  & 0\\
%B(A^{-1}-X^{-1}) & I
%\end{bmatrix}
%\begin{bmatrix}
%A  & -\transpose{B}\\
%0 & S
%\end{bmatrix}\\
%=&
\begin{bmatrix}
X^{-1}  & X^{-1}\transpose{B}T^{-1}\\
0 & T^{-1}
\end{bmatrix}
\begin{bmatrix}
A  & -\transpose{B}\\
B(I -X^{-1}A) & T
\end{bmatrix}\\
=&
\begin{bmatrix}
X^{-1}\left(A + \transpose{B}T^{-1}B(I-X^{-1}A)\right)  & 0\\
T^{-1}B(I -X^{-1}A) & I
\end{bmatrix}
\end{align*}
%
We have
%
\begin{align*}
BX^{-1}\left(A + \transpose{B}T^{-1}B(I-X^{-1}A)\right) = B
\end{align*}
%

%
%~~~~~~~~~~~~~~~~~~~~~~~~~~~~~
\subsubsection{System with constraint on pressure}
%~~~~~~~~~~~~~~~~~~~~~~~~~~~~~
%

%%%%%%%%%%%%%%%%%%%%%%%%%%%%%%%%%%%%%%%%%%%%%%%%%%%%%%%%%
We have to solve (\ref{eq:StokesDirPmean}) with 
%
\begin{align*}
\mathcal A = 
\begin{bmatrix}
A  & -\transpose{B}& 0\\
B & 0 & \transpose{C}\\
0 & C & 0
\end{bmatrix}
=
\begin{bmatrix}
I  & 0& 0\\
BA^{-1} & I & 0\\
0 & CS^{-1} & I
\end{bmatrix}
\begin{bmatrix}
A  & 0& 0\\
0 & S & 0\\
0 & 0 & T
\end{bmatrix}
\begin{bmatrix}
I  & -A^{-1}\transpose{B}& 0\\
0 & I & S^{-1}\transpose{C}\\
0 & 0 & I
\end{bmatrix}
\end{align*}
%
where $S=BA^{-1}\transpose{B}$, $T=-CS^{-1}\transpose{C}$. We have
%
\begin{align*}
\mathcal A^{-1} = 
\begin{bmatrix}
I  & A^{-1}\transpose{B}& 0\\
0 & I & -S^{-1}\transpose{C}\\
0 & 0 & I
\end{bmatrix}
\begin{bmatrix}
A^{-1}  & 0& 0\\
0 & S^{-1} & 0\\
0 & 0 & T^{-1}
\end{bmatrix}
\begin{bmatrix}
I  & 0& 0\\
-BA^{-1} & I & 0\\
0 & -CS^{-1} & I
\end{bmatrix}
\end{align*}
%
We construct our preconditioner by approximations of $A$, $S$, and $T$. The preconditioner $(y_v,_p,y_{\lambda})\to(x_v,x_p,x_{\lambda})$has the steps
%
\begin{align*}
%
\left\{
\begin{aligned}
A {x_v'} =& y_v\\
S {x_p'} =& y_p - B {x_v'}\\
T x_{\lambda} =& y_{\lambda} - C {x_p'}\\
S x_p'' =& \transpose{C} x_{\lambda}\\
{x_p} =& {x_p'} - x_p''\\
A x_v'' =& \transpose{B} x_p\\
x_v =& x_v' + x_v''
\end{aligned}
\right.
%
\end{align*}
%

%
%
%==========================================
\printbibliography[title=References Section~\thesection]
%==========================================
