% !TEX root = ../simfempy.tex
%
%==========================================
\section{Linear elements for the Stokes problem}\label{sec:}
%==========================================
%
Let $\Omega\subset \R^d$, $d=2,3$ be the computational domain. We suppose to have a disjoined partition of its boundary:
$\partial\Omega=\GammaD\cup\GammaN\cup\GammaR$.
%
\begin{equation}\label{eq:}
%
\left\{
\begin{aligned}
-\divv\left(\mu\nabla v\right) + \nabla p = f \quad \mbox{in $\Omega$}\\
\divv v  = g \quad \mbox{in $\Omega$}\\
v = \vD \quad \mbox{in $\GammaD$}\\
\mu\frac{\partial v}{\partial n} + p n = \vN \quad \mbox{in $\GammaN$}\\
c_R v + \mu\frac{\partial v}{\partial n} + p n = \vR \quad \mbox{in $\GammaR$}\\
\end{aligned}
\right.
%
\end{equation}
%
%
%-------------------------------------------------------------------------
\subsection{Implementations of Dirichlet condition}\label{subsec:}
%-------------------------------------------------------------------------
%
We write the discrete velocity space $V_h$ as a direct sum $V_h = \Vint_h \oplus \Vdir_h$, with $\Vdir_h$ corresponding to the discrete functions not vanishing on $\GammaD$. 
Splitting the matrix and right-hand side vector correspondingly, and letting $u^D_h\in \Vdir_h$ be an approximation of the Dirichlet data $\vD$ we have the traditional way to implement Dirichlet boundary conditions:
%
\begin{equation}\label{eq:StokesDirTrad}
\begin{bmatrix}
\bdryint{A} & 0 & -\transpose{\bdryint{B}}\\
0 & I & 0 \\
\bdryint{B} & 0 & 0
\end{bmatrix}
\begin{bmatrix}
\bdryint{v}_h\\
\bdrydir{v}_h\\
p_h
\end{bmatrix}
=
\begin{bmatrix}
\bdryint{f} - \Aintdir v^D_h\\
v^D_h\\
g - \bdrydir{B} v^D_h
\end{bmatrix}.
\end{equation}
%
As for the Poisson problem, we obtain an alternative formulation   
%
\begin{equation}\label{eq:StokesDirNew}
\begin{bmatrix}
\bdryint{A} & 0 & -\transpose{\bdryint{B}}\\
0 & \bdrydir{A} & 0 \\
\bdryint{B} & 0 & 0
\end{bmatrix}
\begin{bmatrix}
\bdryint{v}_h\\
\bdrydir{v}_h\\
p_h
\end{bmatrix}
=
\begin{bmatrix}
\bdryint{f} - \Aintdir v^D_h\\
\bdrydir{A} v^D_h\\
g - \bdrydir{B} v^D_h
\end{bmatrix}.
\end{equation}
%
%
%~~~~~~~~~~~~~~~~~~~~~~~~~~~~~
\subsubsection{Pressure mean}
%~~~~~~~~~~~~~~~~~~~~~~~~~~~~~
%
If all boundary conditions are Dirichlet, the pressure is only determined up to a constant. In order to impose the zero mean on the pressure, let $C$ the matrix of size $(1,nc)$
%
\begin{equation}\label{eq:StokesDirPmean}
\begin{bmatrix}
A  & -\transpose{B}& 0\\
B & 0 & \transpose{C}\\
0 & C & 0
\end{bmatrix}
\begin{bmatrix}
v\\
p\\
\lambda
\end{bmatrix}
=
\begin{bmatrix}
f\\
g\\
0
\end{bmatrix}.
\end{equation}
%
Let us considered solution of (\ref{eq:StokesDirPmean}) with $S=BA^{-1}\transpose{B}$, $T=CS^{-1}\transpose{C}$
%
\begin{equation}\label{eq:}
%
\left\{
\begin{aligned}
&A \tilde v &&= f\\
&S \tilde p &&= g-B\tilde v\\
&T \lambda &&= -C\tilde p\\
&S (p-\tilde p) &&= \transpose{C} \lambda\\
&A(v-\tilde v) &&=  \transpose{B} p
\end{aligned}
\right.
%
\end{equation}
%

%
%
%==========================================
\printbibliography[title=References Section~\thesection]
%==========================================
